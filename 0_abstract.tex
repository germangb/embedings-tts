\chapter*{Abstract}

In this project we attempt to enhance the features of a statistical parametric speech synthesis (SPSS) system based on long short-term memory recurrent neural networks (RNN-LSTM) in order to be able to perform better with an expressive corpus.

The corpus that was used came from the Blizzard challenge \cite{blizzard} and contains roughly 20 hours of audiobooks along with the transcripts. In order to obtain the expressive features (embeddings) we train a convolutional neural network (CNN) to predict how good or bad a sentence is using text from the Stanford Sentiment Treebank \cite{socher2013recursive} dataset, and transfer its mapping function afterwards to a second CNN that works with the waveforms of the Blizzard corpus.

We perform an additional adaptation task to the network that works with text in order to adapt it to the Blizzard corpus, because the two datasets contain different semantic content.

Various experiments have been carried out to find the set of expressive features that give the best objective and subjective results (including a baseline) by means of an evaluation done by TODO volunteers.

TODO results

%\chapter*{Resum}

%\chapter*{Resumen}

\chapter*{Acknowledgements}
